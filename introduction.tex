%!TEX root = paper.tex

\section{Introduction}
\label{sec:intro}

The Livermore Computing (LC) facility at Lawrence Livermore National Laboratory
(LLNL) supports around 2,500 users on 25 different clusters, ranging 
in size from a 1.6 teraflop, 256-core cluster to to the
20 petaflop, 1.6 million-core Sequoia machine, currently ranked first and
third on the Graph500~\cite{graph500} and Top500~\cite{top500}
lists, respectively.
%
%The massively parallel simulations that run on LC machines support
%U.S. Department of Energy missions across a range of scientific domains, 
%including nuclear physics, material science, climate science, mechanical 
%engineering, geophysics, seismology, and stockpile stewardship.
%
The simulation software that runs on these machines is very complex; some
codes depend on specific versions of over 50 dependency libraries.
They require specific compilers, build options and MPI implementations to
achieve the best performance, and users may run several
different codes in the same environment as part of larger 
scientific workflows.

To support the diverse needs of applications, administrators and application
developers 
frequently build, install, and support many different configurations
of math and physics libraries, as well as other software.
Frequently, applications must be rebuilt to fix bugs and to support
new versions of the OS, MPI, compilers, and other dependencies.
Unfortunately, building scientific software is notoriously complex, with
immature build systems that are difficult to adapt to new
machines~\cite{dubois+:comp-sci-eng,hoste+:pyhpc12,wilson+:corr}.
Worse, the space of required builds grows combinatorically
with each new configuration parameter. As a result, LLNL staff
spend countless hours dealing with build and deployment issues.

Existing package management tools automate parts of the build 
process~\cite{bsdports,digirolamo:smithy,dolstra+:icfp08,dolstra+:lisa04,hashdist,homebrew,hoste+:pyhpc12,macports,thiruvathukal:gentoo04}.
For the most part, these systems focus on keeping a single, stable set of 
packages up to date, and they do not handle installation of multiple
versions or configurations.  Those that {\it do} handle multiple configurations
typically require that package files be created for each combination of 
options~ \cite{digirolamo:smithy,dolstra+:icfp08,dolstra+:lisa04,hashdist,hoste+:pyhpc12}, 
leading to a profusion of files and maintenance issues.
To our knowledge, no existing tool allows new configurations to be composed
and assembled on demand.  Some allow limited forms of 
composition~\cite{hoste+:pyhpc12,dolstra+:icfp08,dolstra+:lisa04}, but their
dependency management is overly rigid, and they burden users with
the combinatoric problems of naming and versioning.

In this paper, we describe our experiences with the {\it Spack} package manager,
which we have developed at LLNL to manage increasing software complexity.
Specifically, this paper offers the following contributions:
\begin{enumerate}
\item A novel, recursive syntax for concisely specifying and querying
      the large parameter space of HPC packages;
\item A build methodology that ensures packages find their dependencies
      regardless of users' environments; 
\item Spack: An implementation of these concepts; and
\item Three use cases detailing LLNL's use of Spack to rapidly compose
      software configurations, manage Python installations, and implement
      site-specific build policies.
\end{enumerate}

Spack is in active development at LLNL, and our use cases outline some of
its first applications in our environment.  Spack solves software problems 
that are pervasive at large, multi-user HPC centers, and we believe that our
experiences with it can be applied at other HPC facilities.  Spack improves
operational efficiency by simplifying the build and deployment of 
bleeding-edge scientific software.