%!TEX root = paper.tex

\section{Introduction}
\label{sec:intro}

\todo{1 page}









\begin{verbatim}
- Why do supercomputing centers need this?


- Build space on HPC machines is exponential
	- each new variable multiplies number of build combinations
	- complicated by the ABI.
	- often need a new version on demand
		- cannot just build it.

	- need tools that can sample space.
		
	- users cannot navigate space, and cumbersome to specify all details
	- users often only care about a subset of constraints and just want things
		built.
	- users need to run code again later, but don't remember complex module
		setup -- dynamic settings lead to user support calls and difficulties.
	
	
	- install procedure is different between users and admins
		- duplicated effort
		- need reusable package file.

	
	
	

	- specs
		- describe concisely
		- query installed stuff
		- modules require standardized set of names (no order)

	- flexibility (managing combinatoric configuration space)

	- RPATHs instead of (or in addition to) modules
		- pkgs work as built when run
		- spack means there is no module incantation required

	- code teams can build own libs & stacks

	- Contributions:
		1. Flexible mechanism for specifying and querying config space (spec)
		2. Implementation of this in spack
		3. 3 detailed use cases outlining our experiences at LLNL
			a. combinatoric versions
			b. multi-version python installation
			c. site-specific and user-specific policies
\end{verbatim}

