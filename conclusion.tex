%!TEX root = spack-sc15.tex

\section{Conclusion}
\label{sec:conclusion}


The complexity of managing HPC software is rapidly increasing, and it
will continue unabated without better tools.
In this paper, we reviewed the state of software management tools
across a number of HPC sites, with particular focus on Livermore
Computing (LC). While tools exist that can handle multi-configuration
installs, none of them addresses the combinatorial nature
of the software configuration space directly. None of them allows
a user to {\it compose} new builds with version, compiler,
and dependency parameters rapidly.

We introduced Spack, a package manager in development at LLNL, that
provides truly {\it parameterized} builds.  Spack implements
a novel, recursive {\it spec} syntax that simplifies the process of working
with large software configuration spaces, and it builds software
so that it will run correctly, regardless of the environment.
We outlined a number of Spack's unique features, including
versioned virtual dependencies, and its novel {\it concretization}
process, which converts an abstract build DAG into a concrete,
build-able spec.

We showed through four use cases that Spack is already increasing
operational efficiency in production at LLNL. The
software management techniques implemented in Spack are applicable
to a broad range of HPC facilities.
Spack is available online at \url{http://github.com/scalability-llnl/spack}.
