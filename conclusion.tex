%!TEX root = paper.tex

\section{Conclusion}
\label{sec:conclusion}


The complexity of managing HPC software is rapidly increasing, and it
will continue unabated without effective tools.
In this paper, we reviewed the state of software management tools 
across a number of HPC sites, with particular focus on Livermore
Computing (LC). While tools exist that can handle multi-configuration
installs, none of them addresses the combinatorial nature
of the software configuration space directly. None of them allows
a user to rapidly {\it compose} new parametric builds.

We introduced Spack, a package manager in development at LLNL, that
provides truly {\it parameterized} builds.  Spack implements
a novel, recursive {\it spec} syntax that simplifies the process of working
with large software configuration spaces, and it builds software 
so that it will run correctly, regardless of the environment.
We outlined a number of additional unique features of Spack, including
versioned virtual dependencies, and Spack's {\it concretization} process,
which converts an underspecified build DAG into a buildable spec.
We showed through several use cases that Spack is already
increasing operational efficiency and simplifying software management
at LLNL, and that it can be rapidly adapted to new production use cases.



