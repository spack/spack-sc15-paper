%!TEX root = spack-sc15.tex

\subsection{Combinatorial Naming}
\label{sec:usecase-combinatorial}
%\subsubsection{Gperftools and MPILeaks}

{\tt Gperftools} is a suite of tools from Google that has gained popularity among
developers for its high-performance thread-safe heap and its lightweight profilers.
Unfortunately, two issues made it difficult to maintain {\tt gperftools} installations
at LC.  First, {\tt gperftools} is a C++ library.  Since C++ does not define a standard
ABI, {\tt gperftools} must be rebuilt with each compiler and compiler version used by client applications.  Second, building {\tt gperftools} on bleeding-edge architectures
(such as Blue Gene/Q) requires patches and complicated configure lines that
change with each compiler.  One application team tried to maintain their own
builds of {\tt gperftools}, but the maintenance burden soon became too great.

\begin{figure}
\begin{minted}[linenos,
               numbersep=5pt,
			   fontsize=\scriptsize,
               frame=lines,
               framesep=2mm]{python}
patch('patch.gpeftools2.4_xlc', when='@2.4 %xlc')

def install(self, spec, prefix):
    if spec.architecture == 'bgq' and self.compiler.satisfies('xlc'):
        configure("--prefix=" + prefix,  "LDFLAGS=-qnostaticlink")
    elif spec.architecture == 'bgq':
        configure("--prefix=" + prefix,  "LDFLAGS=-dynamic")
    else:
        configure("--prefix=" + prefix)

    make()
    make("install")
\end{minted}
  \caption{
    Simplified install routine for gperftools.
    \label{fig:gperftools}
  }
\end{figure}

Spack presented a solution to both problems.  Package administrators can use Spack to
maintain a central install of {\tt gperftools} across combinations of
compilers and compiler versions easily.  Spack's {\tt gperftools} package 
also serves as a central
institutional knowledge repository, as its package files encode
the patches and configure lines required for each platform and compiler combination.
%
Figure~\ref{fig:gperftools} illustrates per-compiler and platform build rules with
a simplified version of the install routine for {\tt gperftools}.
%(the full {\tt gperftools} install routine includes other compilers and more options).
The package applies a patch if {\tt gperftools} 2.4 is built with the XL compiler,
and it selects the correct configure line based on the spec's platform and compiler.

LC's {\tt mpileaks} tool, which has been a running example in this paper,
has an even larger configuration space than {\tt gperftools}.  It must be built
for different compilers, compiler versions, MPI implementations, and MPI
versions. As with {\tt gperftools}, we have been able to install many different
configurations of {\tt mpileaks} using Spack.  Moreover, Spack's virtual dependency
system allows us to compose a new {\tt mpileaks} build quickly when a new MPI library is
deployed at LC, {\it without} modifying the {\tt mpileaks} package itself.
%Though we simplified its dependency structure
%here, the {\tt callpath} library can be built with three different stack trace libraries:
%{\tt dyninst}, {\tt libunwind}, or {\tt glibc}.  For some codes we may soon need to build
%different versions of {\tt mpileaks} with different {\tt callpath} configurations.
