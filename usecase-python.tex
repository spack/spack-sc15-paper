%!TEX root = spack-sc15.tex

\subsection{Support for interpreted languages}
\label{sec:usecase-python}

Python is becoming increasingly popular for HPC applications,
due to its flexibility as a language and its excellent support
for calling into fast, compiled numerical libraries.
Python is an interpreted language, but one can use it
as a friendlier interface to compiled libraries like FFTW, ATLAS, and
other linear algebra libraries.  Many LLNL code teams use Python in this manner.
%
LC supports Python installations for several application teams,
and maintaining these repositories has grown increasingly complex over
time. The problems are similar to those that drove us to create
Spack: different teams want different Python libraries with different
configurations.

Existing Python package managers are either language-specific~\cite{eby:setuptools},
or they do not support building from source~\cite{anaconda,conda}. None
handles multi-configuration builds.  More glaringly, Python extensions
are usually installed into the Python interpreter's prefix.
This makes it impossible to install multiple versions.\footnote{{\tt setuptools}
has support for multiple versions via {\tt pkg_resources},
but this requires modifications to client code.}
Installing each extension in its own prefix enables combinatorial versioning,
but it requires users to add packages to the {\tt PYTHONPATH} variable at runtime.

Per Spack's design philosophy, we wanted a way to easily manage many different
versions, but {\it also} to provide a baseline set of extensions {\it without}
requiring environment settings.
%
To support this mode of operation, we added the concept of {\tt extension} packages
to Spack. Python modules use the {\tt extends('python')} directive instead of
{\tt depends\_on('python')}.
Each module installs into its own prefix like any other package,
and each module depends on a particular Python installation.
But extensions can be {\tt activated} or {\tt deactivated} an
within the dependent Python installation.  The {\tt activate} operation
symbolically links each file in the extension prefix into the Python
installation prefix, as though it were installed directly. If any file
conflict would arise from this operation, {\tt activate} fails.
Similarly, the {\tt deactivate} operation removes the symbolic links and restores
the Python installation to its  pristine state.

There were additional complications because many Python packages {\it install their own
package manager} if they do not find one in the Python installation.
There are also many ways that Python packages add themselves to the
interpreter's default path, and some of them conflict. We modified
Spack so that extendable packages, like Python, can supply custom code
in the package file that that specializes {\tt activate} and {\tt deactivate}
for the particular package. Python uses this feature to merge conflicting
files during activation.  The end result is that Python extensions can
be installed automatically in their own prefixes, and they can be composed
with a wide range of bleeding-edge libraries that other package managers do
not handle.

%To experiment with these extensions, users can load environment
%modules generated by Spack. If they want a particular version to be available
%without any special environment settings, they can activate it within the Python instance.

Spack essentially implements a ``meta package-manager'' for each Python
instance, which can coexist with Spack's normal installation model.
This has allowed us to efficiently support our application teams, for whom we can
now rapidly construct custom Python installations.  We have also reduced
the amount of time that LC staff spend installing Python modules.
Because Spack packages can extend the activation and deactivation mechanisms,
we believe the same mechanism could be used with other
interpreted languages with similar extension models, such as R, Ruby, or Lua.
