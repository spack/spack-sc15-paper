%!TEX root = spack-sc15.tex

\section{The Spack Package Manager}
\label{sec:implementation}
Based on our experiences at LLNL, we have developed
{\it Spack}, the Supercomputing PACKage manager.
Spack is written in Python.  We chose Python for its flexibility
and its wide use in the HPC community.
%
Like prior systems, Spack supports an arbitrary number of software
installations, and like Nix it can identify them with hashes.  Unlike any
prior system, Spack provides a language to specify and manage the
combinatorial space of HPC software configurations.

\noindent
Spack provides the following unique features:
\begin{enumerate}
\item Packages explicitly {\bf parameterized} by version, platform,
      compiler, options, and dependencies, to allow easy composition.
\item A novel, recursive {\bf spec syntax} for dependency graphs and constraints,
      which aids in managing the build parameter space.
\item {\bf Versioned virtual dependencies} to handle versioned, 
      ABI-incompatible interfaces like MPI.
\item A novel {\bf concretization} process that translates an abstract build
      specification into a full, concrete build specification.
\item A build environment with {\bf compiler wrappers} that enforce build
      consistency and simplify package writing.
\end{enumerate}

%!TEX root = spack-sc15.tex

\subsection{Packages}\label{sec:packages}

%\subsubsection{Package Files}
\begin{figure}
\begin{minted}[linenos,
               numbersep=5pt,
			   fontsize=\scriptsize,
               frame=lines,
               framesep=2mm]{python}
from spack import *

class Mpileaks(Package):
    """Tool to detect and report leaked MPI objects like
       MPI_Requests and MPI_Datatypes."""

    homepage = "https://github.com/hpc/mpileaks"
    url = homepage + "/releases/download/v1.0/mpileaks-1.0.tar.gz"

    version('1.0', '8838c574b39202a57d7c2d68692718aa')
    version('1.1', '4282eddb08ad8d36df15b06d4be38bcb')

    depends_on("mpi")
    depends_on("callpath")

    def install(self, spec, prefix):
        configure("--prefix=" + prefix,
                  "--with-callpath=" + spec['callpath'].prefix)
         make()
         make("install")
\end{minted}
	\caption{
		Spack package for the {\tt mpileaks} tool.
		\label{fig:mpileaks}
	}
\end{figure}

In Spack, packages are Python scripts that build software artifacts.
Each package is a class that extends a generic {\tt Package}
base class.  {\tt Package} implements the bulk of the build process, but
subclasses provide their own {\tt install} method to handle the
specifics of particular packages. The subclass does {\it not} have to
handle managing the install location.  Rather, Spack passes the {\tt install}
method a {\tt prefix} parameter.  The package implementor must ensure that
her {\tt install} function installs the package {\it into} the {\tt prefix},
but Spack ensures that {\tt prefix} is computed in such a way that it is
unique for every configuration of a package.  To further simplify
package implementation, Spack implements an embedded domain-specific
language (DSL).
The DSL provides special directives such as {\tt depends\_on},
{\tt version}, {\tt patch}, and {\tt provides} that add metadata
to the package class.

Figure~\ref{fig:mpileaks} shows the package for \mpileaks, a tool developed
at LLNL to find un-released MPI handle objects in parallel programs.
The {\tt MpiLeaks} class provides simple metadata on lines 4-8: a text
description, a homepage, and a download URL.
Two {\tt version} directives on lines 10-11 identify known versions and provide
MD5 checksums to verify downloads.
The two {\tt depends\_on} directives on lines 13-14 indicate prerequisite
packages that must be installed before \mpileaks.
Last, the {\tt install} method on line 16 contains the commands for building.
Spack's DSL allows shell commands to be invoked as Python functions,
and the {\tt install} method invokes {\tt configure},
{\tt make}, and {\tt make install} as a shell script would.

%!TEX root = paper.tex

\begin{figure}
	\begin{subfigure}{\linewidth}
		\centering
		\includegraphics[width=\columnwidth]{specs/mpileaks.pdf}
		\caption{
			Spec for {\tt mpileaks}
			\label{fig:specs-mpileaks}
		}
	\end{subfigure}
%
	\begin{subfigure}{\linewidth}
		\centering
		\includegraphics[width=\columnwidth]{specs/mpileaks-version}
		\caption{
			{\tt mpileaks@2.3}
			\label{fig:specs-mpileaks-version}
		}
	\end{subfigure}
%
	\begin{subfigure}{\linewidth}
		\centering
		\includegraphics[width=\columnwidth]{specs/mpileaks-abstract.pdf}
		\caption{
			{\tt mpileaks@2.3 \^{}callpath@1.0+debug \^{}libelf@0.8.11}
			\label{fig:specs-mpileaks-abstract}
		}
	\end{subfigure}
%
	\caption{
		Constraints applied to {\tt mpileaks} specs.
	}
\end{figure}



\subsection{Spack Specs}\label{sec:specs}

Using the simple script in Figure~\ref{fig:mpileaks}, Spack can build many different
versions and configurations of the {\tt mpileaks} package.  In traditional port systems,
package code is structured to build a single version of a package, but in Spack, each
package file is a {\it template} that can be built in many different ways.  The build is
{\it parameterized} so that it can be configured many different ways.
Spack calls a single build configuration a {\it spec},
 and the {\tt spec} object passed to {\tt install()}
is how the Spack system encapsulates build parameters for package authors.

\subsubsection{Structure}
To understand how specs work, consider the {\tt mpileaks} package structure.
Metadata in the {\tt Mpileaks} class ({\tt version}, {\tt depends\_on}, etc.) describe
its relationships with other packages.  There are two direct dependencies:
the {\tt callpath} library and {\tt mpi}.  Spack recursively inspects the class definitions
for each dependency and constructs a graph of their relationships.  The result
is a directed, acyclic graph (DAG) of dependencies\footnote{Spack currently disallows
circular dependencies.}.
%
To guarantee a consistent build, and to avoid ABI incompatibility, the DAG
is constructed so that there is only {\it one} version of any particular package.  Note
the distinction: while Spack can install arbitrarily many configurations of any package,
no two configurations of the same package will ever appear in the same build DAG.

DAGs for {\tt mpileaks} are shown in Figure~\ref{fig:specs}.
Each node is a package and has five configuration parameters that control
how it will be built: 1) the package version, 2) the compiler to
build with, 3) the compiler version, 4) named compile-time build options, or {\it variants},
and 5) the target architecture.


\subsubsection{Configuration Complexity}
A spec DAG has many degrees of freedom, and it is not reasonable to expect users to
understand or specify all of them.  In our experience at LLNL, the typical user
only cares about a small number of build constraints (if any), and does not know enough to
specify the rest. For example, a user may know that a certain version of a library like
{\tt boost} is required, but only cares that other build parameters are set so that
the build will succeed.
%
Configuration complexity makes the HPC software ecosystem difficult to manage: there are
simply too many parameters. However, the small set of important build constraints can be
very specific, so we have two competing concerns.  We need the ability to specify details
of the configuration space, without the complexity of remembering all of them.

\subsubsection{Spec Syntax}\label{sec:syntax}

\begin{figure}
\begin{grammar}
  <spec>         ::= <id> [ constraints ]

  <constraints>   ::= \{ `@' <version-list> | `+' <variant> \newline
                   | `-' <variant> ~~~| `~' <variant> \newline
                   | `\%' <compiler> ~| `=' <architecture> \} \newline
                  [ <dep-list> ]

  <dep-list>  ::= \{ `\textsf \textasciicircum' <spec> \}

  <version-list> ::= <version> [ \{ `,' <version> \} ]

  <version>      ::= <id> | <id> `:' | `:' <id> | <id> `:' <id>

  <compiler>     ::= <id> [ <version-list> ]

  <variant>      ::= <id>

  <architecture> ::= <id>

  <id>           ::= [A-Za-z0-9_][A-Za-z0-9_.-]*
\end{grammar}
\caption{
	EBNF grammar for spec expressions.
	\label{fig:grammar}
}
\end{figure}


\begin{table*}\centering
\begin{tabular}{|r|p{2.4in}|p{4in}|}
\hline
& {\bf Spec} & {\bf Meaning} \\
\hline
\hline
1&\small\verb|mpileaks|                         & \small {\tt mpileaks} package, no constraints. \\\hline
2&\small\verb|mpileaks@1.1.2|                   & \small {\tt mpileaks} package, version 1.1.2. \\\hline
3&\small\verb|mpileaks@1.1.2 %gcc|              & \small {\tt mpileaks} package, version 1.1.2, built with {\tt gcc} at the default version. \\\hline
4&\small\verb|mpileaks@1.1.2 %intel@14.1 +debug| & \small {\tt mpileaks} package, version 1.1.2, built with Intel compiler version 14.1, \newline with the ``debug'' build option. \\\hline
5&\small\verb|mpileaks@1.1.2 =bgq|              & \small {\tt mpileaks} package, version 1.1.2, built for the Blue Gene/Q platform (BG/Q). \\\hline
6&\small\verb|mpileaks@1.1.2 ^mvapich2@1.9|     & \small {\tt mpileaks} package version 1.1.2, using {\tt mvapich2}  version 1.9 for MPI. \\\hline
7&\small\verb|mpileaks @1.2:1.4 %gcc@4.7.5 -debug =bgq| \newline
      \verb|  ^callpath @1.1 %gcc@4.7.2| \newline
      \verb|  ^openmpi @1.4.7|                & \small%
      {\tt mpileaks} at any version between 1.2 and 1.4 (inclusive), built with gcc 4.7.5,
      without the debug option, for BG/Q, linked with {\tt callpath} version 1.1
      and building {\tt callpath} with {\tt gcc} version 4.7.2, linked with {\tt openmpi} version 1.4.7.    \\
\hline
\end{tabular}
\caption{
	Spack build spec syntax examples and their meaning.
	\label{tab:specs}
}
\end{table*}


We have developed a syntax for specs that allows users to specify
only those constraints they care about, but to be specific when necessary.
Our syntax is expressive enough to represent software DAGs  but concise enough to use
on the command line. The spec syntax is recursively defined to allow users to specify
parameters on dependencies as well as on the root of the DAG.  The EBNF grammar we use
to implement this in Spack is shown in Figure~\ref{fig:grammar}.

We begin with a simple example.
Consider the case where a user wants to install the {\tt mpileaks} package, but knows
nothing about its structure.  To install the package, the user invokes the {\tt spack install}
command:
%
\begin{minted}[fontsize=\scriptsize]{bash}
    $ spack install mpileaks
\end{minted}
%
Here, {\tt mpileaks} is the simplest possible spec---a single identifier.
Spack parses it and converts it into the DAG shown in Figure~\ref{fig:specs-mpileaks}.
Note that even though the spec contains no dependency information, it is still
converted to a full DAG, based on the directives supplied in package files. Since there
are no constraints on the nodes, Spack has considerable leeway for how to build the package,
and we say that the package is {\it unconstrained}.
%
Now, suppose the user wants a specific version of {\tt mpileaks}.  This can be requested
with a version constraint after the package name:
%
\begin{minted}[fontsize=\scriptsize]{bash}
    $ spack install mpileaks@2.3
\end{minted}
%
We see from Figure~\ref{fig:specs-mpileaks-version} that the specific version constraint is
placed on the {\tt mpileaks} node in the DAG, but the rest of the DAG remains unconstrained.
If the user does not need a specific version of {\tt mpileaks}, but does require
particular minimum version, then the user could use {\it version range} syntax.
and write {\tt @2.3:}.  Likewise, for a version between 2.3 and 2.5.6, she would use
{\tt @2.3:2.5.6} to designate the range. In these cases, the user can save build time if
Spack already has a version installed that satisfies the constraint -- Spack will just use
the previously-built installation instead of building a new one.

Figure~\ref{fig:specs-mpileaks-abstract} shows the recursive nature of the spec syntax:
%
\begin{minted}[fontsize=\scriptsize]{bash}
    $ spack install mpileaks@2.3 ^callpath@1.0+debug ^libelf@0.8.11
\end{minted}
%
The caret (\verb|^|) denotes constraints for a particular dependency.  In the DAG,
we now see that there are version constraints on {\tt callpath} and {\tt libelf},
and that the user has asked for the debug variant of the {\tt callpath} library.

Recall that Spack guarantees there will be only a single version of any package in
the spec DAG.  Therefore, within the same DAG, each dependency can be uniquely identified by
only its package name.  The user does not have to think about DAG connectivity to add
constraints.  She need only know that a package depends, somehow, on {\tt callpath}.
For the same reason, the constraint order does not matter; dependency constraints
can appear in arbitrary order.

Table~\ref{tab:specs} shows further examples of specs, ranging from very
simple to more complex. From these examples, we can see that Spack offers constraint
notation to cover the rest of the HPC package parameter space.

{\bf Versions.}
The version constraint, denoted with {\tt @}, was already covered above. Versions
can be precise ({\tt @2.5.1}) or denote a potentially open-ended
range ({\tt @2.5:}, {\tt @2.5:4.4}).

The package in Figure~\ref{fig:mpileaks} lists two ``safe'' versions with checksums, but
in our experience users frequently want bleeding-edge versions.  Package managers
frequently lag behind the latest releases.
Spack has a capability to extrapolate URLs from versions,
using the package's {\tt url} attribute as a model\footnote{This works
for packages with consistently named URLs}.  The user can request a specific
version on the command line, even if it is unknown to Spack,
and Spack will attempt to install it.  Spack also uses the same
model to scrape webpages and find new versions as they become available.

{\bf Compilers.}
With a compiler constraint (shown on line 3) the user
simply adds {\tt \%} followed by its name, along with an optional compiler version
specifier.  Spack compiler names, e.g. {\tt gcc}, refer to the full compiler {\it toolchain},
i.e. the C, C++, Fortran 77, and Fortran 90 compilers.  Spack can auto-detect
compiler toolchains if they are in the user's {\tt PATH} or they can be registered manually
through a configuration file.

{\bf Variants.}
To handle build options like compiler flags or optional components, specs can
have named flags, or {\it variants}.  Variants are associated with the package,
so the {\tt mpileaks} package implementor must check the spec and handle the cases
where debug is enabled ({\tt +debug}) and disabled (with {\tt -debug}
or {\tt ~debug}).  The names simplify the versioning and prevent
Spack's configuration space from becoming too fine-grained.
It would violate our goal of conciseness, for example, to include detailed
compiler flags in spec syntax, but known sets of flags can simply be named.

{\bf Cross-compilation.}
To support cross-compilation, Spack includes the platform in the package spec (line 5).
Platforms begin with {\tt =} and take names like {\tt linux-ppc64} or {\tt bgq}.  They are
specified per-package; this allows front-end tools to depend on their back-end measurement
libraries with a {\it different} architecture on cross-compiled machines.

\subsubsection{Constraints in packages}

So far, we have shown examples of specs being used to request constraints from the
command line, when {\tt spack install} is invoked.  However, the user is not the only
source of constraints.  Applications may require specific versions of dependencies,
and it is often desirable to write constraints like this into a package file.  For
example, the ROSE compiler only builds with a certain version of the {\tt boost} library.
At first glance, the {\tt depends\_on()} directives in Figure~\ref{fig:mpileaks} look
like they take simple package names.  However, a package name is also a spec, and
the same constraint syntax usable from the command line can be applied inside directives.
So, the ROSE compiler can simply write:
%
\begin{minted}[fontsize=\scriptsize]{python}
    depends_on('boost@1.54.0')
\end{minted}
%
This constraint will be incorporated into the initial DAG node generated from
the ROSE package.

Spack's {\tt patch} directive also accepts spec syntax as a predicate in an
optional {\tt when} parameter.  For example, these concise directives in the
Python package ensure that specific patches are applied to the Python source
code when it is being built on Blue Gene/Q, with the appropriate compiler:
%
\begin{minted}[fontsize=\scriptsize]{python}
    patch('python-bgq-xlc.patch',   when='=bgq%xlc')
    patch('python-bgq-clang.patch', when='=bgq%clang')
\end{minted}
%
Constraints in the {\tt when} clause are matched against the Python package spec.
Outside of directives, constraints can be used directly
with the {\tt spec} object in the {\tt install} method:
%
\begin{minted}[fontsize=\scriptsize]{python}
  def install(self, spec, prefix):
      if spec.satisfies('%gcc'):
          # Handle gcc
      elif spec.satisfies('%xlc'):
          # Handle XL compilers
      ...
\end{minted}
%

\subsubsection{Build specialization}

In our experience maintaining packages at LLNL, we have sometimes had to change
entire package build scripts due to large changes in the way certain packages build.
This can be very cumbersome, and it is difficult to maintain both the old and new
version of a build script, but we must if we want to keep installing older versions
for users who rely on them.


%class Dyninst(Package):
%    """"Dyninst binary instrumentation and analysis tool."""
%    url="http://www.paradyn.org/release8.1/DyninstAPI-8.1.1.tgz"
%    version('8.1.1', 'd1a04e995b7aa70960cd1d1fac8bd6ac')
%    depends_on("libelf")
%    depends_on("libdwarf")
%    depends_on("boost@1.42:")
%
\begin{figure}
\begin{minted}[linenos,
               numbersep=5pt,
			   fontsize=\scriptsize,
               frame=lines,
               framesep=2mm]{python}
    def install(self, spec, prefix):  # default build uses cmake
        with working_dir('spack-build', create=True):
            cmake('..', *std_cmake_args)
            make()
            make("install")

    @when('@:8.1')                    # <= 8.1 uses autotools
    def install(self, spec, prefix):
        configure("--prefix=" + prefix)
        make()
        make("install")
\end{minted}
\caption{
    Specialized {\tt install} method in Dyninst.
	\label{fig:specialization}
}
\end{figure}

For cases like this, Spack provides functionality that allows Python functions to have
multiple definitions, with some specialized for particular configurations of the package.
This allows us to have two separate implementations of {\tt install} or {\it any} method
in a package class. Figure~\ref{fig:specialization} shows how this is used in the Dyninst
package.  The {\tt @when} directive is a Python decorator: a higher order function that
takes a function definition as a parameter and returns a new function to put in its place.
We replace the function with a callable multi-function dispatch object, and we
integrate the predicate check into the function dispatch mechanism.  Here, the
{\tt when} condition is true when Dyninst is at version 8.1 or lower, and in those cases
the package will use the {\tt configure}-based build.  By default if no predicate matches,
{\tt install} will use the default CMake-based implementation. The simple {\tt @when}
annotation allows us to maintain our old build code alongside the new version without
accumulating complex logic in a single {\tt install} function.

%!TEX root = paper.tex


\subsection{Versioned Virtual Dependencies}
	\todo{.5 page}

%!TEX root = paper.tex

\subsection{Abstract Specs \& Concretization}
	\todo{.75 page}
	

Before it builds a package, Spack ensures the following conditions:
\newline

\begin{tabular}{l}
(1) All dependencies are present in the DAG. \\
(2) Parameters are set on all packages in the DAG. \\
\end{tabular}\newline

\noindent
The install process then constructs a package object for each node in the spec DAG
and traverses the DAG in a bottom-up fashion.  At each node, it invokes the package's
{\tt install()} method, using a sub-DAG rooted at the package to be installed as the {\tt spec}
parameter to {\tt install()}. Package maintainers are responsible for examining the spec's
configuration and adjusting the build if necessary.


In our experience, much complexity in build scripts arises from the need for the script to
query the environment and determine how to build.  In spack, build scripts are free from this 
concern, because the package author is guaranteed that a package's build spec is concrete
by the time {\tt install()} is called. There is thus no need to perform complicated state checks,
just to query the spec for its configuration.



%!TEX root = paper.tex


\subsection{Installation Environment}

%{\bf Reproducibility.}
%\paragraph{Reproducibility}
Spack is intended to build a consistent HPC stack for our multi-user
environment, and reproducible builds are one of our design goals.
Experience at LLNL has shown that it is vexingly difficult to reproduce
a build manually.
%
Many packages used at LLNL have a profusion of build options, and specifying them 
correctly often requires tedious experimentation.  This is due to lack of
build standards and to the diversity of HPC environments.  
For example, in some packages that depend on the {\tt Silo} library,
the {\tt ---with-silo} parameter takes a path to {\tt Silo}'s installation prefix.
In others, it takes the {\tt include} and {\tt lib} subdirectories,
separated by a comma.
The {\tt install()} method in Spack's package files allows us to record
precise build incantations for later reuse.

%{\bf Environment isolation.}
\paragraph{Environment isolation}
In addition to command-line issues, we
frequently encounter errors due to inconsistencies between the environment of
the package installer and the package user.
%
For example, there are two versions of the {\tt libelf} library used by
LLNL performance tools. One is distributed with RedHat Linux, while another
publicly available version has the same API but an incompatible ABI.
Failure to specify the right version at build time has caused many
inexplicable crashes.
%
Spack manages the build environment by running the {\tt install} invocation
in a new process.  It helps packages find dependencies 
correctly, by setting
{\tt PATH}, {\tt PKG\_CONFIG\_PATH}, {\tt CMAKE\_PREFIX\_PATH}, and
{\tt LD\_LIBRARY\_PATH} to include the dependencies of the current build.
These variables are commonly used by build systems to locate dependencies,
and setting them helps to ensure that incorrect libraries are detected.
The isolated build environment also gives package authors 
free reign to set build-specific environment variables without interfering
with other packages.


%{\bf Compiler wrappers and RPATHs.}
\paragraph{Compiler wrappers and RPATHs}
Finding compilers at build time is not the only obstacle to reproducible
behavior.  As mentioned in Section~\ref{sec:motivation}, it is also important
for binaries to be able to find dependency libraries at {\it runtime}.
One of the most frequent user errors at LC is improper library configuration.
Users frequently do not know what libraries a package was built with, and 
it is difficult for them to construct a suitable {\tt LD\_LIBRARY\_PATH} for
a package that was built by someone else.  Because of frequent support calls,
we typically add {\tt RPATHs} to public software installations, so that paths
to dependencies are embedded in binaries and so that users do not have to know
this information to run installed software correctly.

Spack manages {\tt RPATH} settings and other build policies with
{\it compiler wrappers}. 
In each isolated {\tt install} environment, Spack sets the standard 
environment variables
{\tt CC}, {\tt CXX}, {\tt F77}, and {\tt FC} to point to its own compiler
wrapper scripts.  These variables are used by most build systems to select
C, C++, and Fortran compilers, so they are generally picked up 
automatically\footnote{If builds do not respect {\tt CC}, {\tt CXX}, etc.,
wrappers can typically be added as arguments or inserted into Makefiles
by {\tt install}.}.
When run, the wrappers insert include ({\tt -I}), library ({\tt -L}), and 
{\tt RPATH} ({\tt -Wl,-rpath} or similar) flags into the argument list.
These point to the {\tt include} and {\tt lib} directories of dependency
library installations, where needed headers and libraries are typically located.
After the modified argument list is constructed, the wrappers delegate it
to the real compiler to execute.

Spack's compiler wrappers have a number of useful effects.  First, they allow
Spack to transparently parametrize the compiler for most builds and for
Spack users to easily request a build with a new compiler swapped in.
Second, they enforce the use of {\tt RPATHs} in
installed binaries.  This causes applications built by Spack to run correctly
{\it regardless of the environment}.  Third, because compiler wrappers add 
header and library search paths for dependencies, header and library detection
tests run by most build systems succeed automatically, {\it without}
the need to use special arguments for nonstandard locations.  {\tt configure}
commands in Spack's {\tt install} function can have fewer arguments, and can
be written as they would be for system installs.  This reduces complexity
for package maintainers and enforces consistent, reproducible
build policies across packages.  Finally, because Spack has control over the 
wrappers, package authors can programmatically filter the compiler flags
used by software build systems, a useful last resort when porting to
bleeding-edge platforms or new, esoteric compilers.

\subsubsection{Environment Module Integration}
\label{sec:envmodule}
In addition to build-time environments, Spack can also assist in create run-time
environments.  A package may need environment variables like {\tt PATH}, 
{\tt LD_LIBRARY_PATH}, or {\tt MANPATH} set before they can be used.  Some users
may prefer to manually point environment variables and paths at packages, and 
Section~\ref{sec:package-views} discusses how Spack can assist those users.  
As discussed in Section~\ref{sec:motivation}, many sites rely on environment 
modules to set their runtime environment.  Spack can automatically create 
dotkit~\cite{dotkit} and Module configuration files for its packages, allowing 
users to setup their runtime environment using familiar systems.  

Future versions of Spack may also allow the creation of Lmod~\cite{mclay:lmod} 
module files.  Lmod allows users to load packages using a heirarchy of module 
load operations.  For example, a user might load a {\tt gcc} module and a 
{\tt mvapich2} module, and a subsequent load of the {\tt MPILeaks} package would 
load the appropriate version for that compiler and MPI implementation.  Spack 
understands where packages would fit in the Lmod hierarchy, allowing it to create 
Lmod module files.  



