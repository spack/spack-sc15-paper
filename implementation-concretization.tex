%!TEX root = paper.tex

\subsection{Abstract Specs \& Concretization}
	\todo{.75 page}
	

Before it builds a package, Spack ensures the following conditions:
\newline

\begin{tabular}{l}
(1) All dependencies are present in the DAG. \\
(2) Parameters are set on all packages in the DAG. \\
\end{tabular}\newline

\noindent
The install process then constructs a package object for each node in the spec DAG
and traverses the DAG in a bottom-up fashion.  At each node, it invokes the package's
{\tt install()} method, using a sub-DAG rooted at the package to be installed as the {\tt spec}
parameter to {\tt install()}. Package maintainers are responsible for examining the spec's
configuration and adjusting the build if necessary.


In our experience, much complexity in build scripts arises from the need for the script to
query the environment and determine how to build.  In spack, build scripts are free from this 
concern, because the package author is guaranteed that a package's build spec is concrete
by the time {\tt install()} is called. There is thus no need to perform complicated state checks,
just to query the spec for its configuration.


