%!TEX root = spack-sc15.tex

\subsection{Packages}\label{sec:packages}

%\subsubsection{Package Files}
\begin{figure}
\begin{minted}[linenos,
               numbersep=5pt,
			   fontsize=\scriptsize,
               frame=lines,
               framesep=2mm]{python}
from spack import *

class Mpileaks(Package):
    """Tool to detect and report leaked MPI objects like
       MPI_Requests and MPI_Datatypes."""

    homepage = "https://github.com/hpc/mpileaks"
    url = homepage + "/releases/download/v1.0/mpileaks-1.0.tar.gz"

    version('1.0', '8838c574b39202a57d7c2d68692718aa')
    version('1.1', '4282eddb08ad8d36df15b06d4be38bcb')

    depends_on("mpi")
    depends_on("callpath")

    def install(self, spec, prefix):
        configure("--prefix=" + prefix,
                  "--with-callpath=" + spec['callpath'].prefix)
         make()
         make("install")
\end{minted}
	\caption{
		Spack package for the {\tt mpileaks} tool.
		\label{fig:mpileaks}
	}
\end{figure}

In Spack, packages are Python scripts that build software artifacts.
Each package is a class that extends a generic {\tt Package}
base class.  {\tt Package} implements the bulk of the build process, but
subclasses provide their own {\tt install} method to handle the
specifics of particular packages. The subclass does {\it not} have to
handle managing the install location.  Rather, Spack passes the {\tt install}
method a {\tt prefix} parameter.  The package implementor must ensure that
her {\tt install} function installs the package {\it into} the {\tt prefix},
but Spack ensures that {\tt prefix} is computed in such a way that it is
unique for every configuration of a package.  To further simplify
package implementation, Spack implements an embedded domain-specific
language (DSL).
The DSL provides special directives such as {\tt depends\_on},
{\tt version}, {\tt patch}, and {\tt provides} that add metadata
to the package class.

Figure~\ref{fig:mpileaks} shows the package for \mpileaks, a tool developed
at LLNL to find un-released MPI handle objects in parallel programs.
The {\tt MpiLeaks} class provides simple metadata on lines 4-8: a text
description, a homepage, and a download URL.
Two {\tt version} directives on lines 10-11 identify known versions and provide
MD5 checksums to verify downloads.
The two {\tt depends\_on} directives on lines 13-14 indicate prerequisite
packages that must be installed before \mpileaks.
Last, the {\tt install} method on line 16 contains the commands for building.
Spack's DSL allows shell commands to be invoked as Python functions,
and the {\tt install} method invokes {\tt configure},
{\tt make}, and {\tt make install} as a shell script would.
