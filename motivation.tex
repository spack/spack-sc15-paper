%!TEX root = paper.tex

\section{Common Practice}
\label{sec:motivation}

This section describes existing tools for managing HPC 
software and motivates the need for a system like Spack.

\paragraph{Meta-build Systems}

{\it Meta-build systems} such as Contractor, WAF, and
MixDown~\cite{amundson:contractor,epperly+:mixdown,nagy:waf} are closely
related to package managers, but their focus is on ensuring that a single
project builds correctly along with all of its dependencies.  They do not address
larger, multi-user system administration issues such as versioning, naming
conventions, rapid multi-configuration installation, or maintaining many
divergent software stacks.

\paragraph{Traditional package managers}
Package managers automate the installation of complex sets of software packages.
{\it Binary package managers} such as RPM, yum, and  
APT~\cite{foster+:rpm03,silva:apt01,yum} are integrated with most
operating system (OS) distributions, and they are used to ensure that dependencies
are installed before packages that require them.
These systems largely solve the problem of managing a single software
stack, which works well for the baseline OS and drivers, which are 
common to all applications on a system.
However, they make the assumption that there is only a single version
of each package, and most install packages in a single, inflexible location.
To install multiple configurations, the user must create a custom, combinatoric
naming scheme to avoid conflicts. They typically require root 
privileges and do not optimize for specific hardware.

{\it Port systems} such as Gentoo, BSD Ports, MacPorts, and 
Homebrew~\cite{bsdports,groffen:gentoo-prefix,homebrew,macports,thiruvathukal:gentoo04}
allow packages to be built on demand, but most suffer from 
the same versioning and naming issues as traditional package managers.
Some allow multiple versions to be installed in the same
prefix~\cite{groffen:gentoo-prefix}, but again the burden is on package
creators to manage conflicts. This restricts installations to only a few
configurations.


\paragraph{Virtual Machines and Containers}

Packaging problems arise in HPC because a supercomputer's hardware, OS, and
filesystem are shared by many users with different requirements.  The classic solution
to this problem is to use virtual
machines (VMs)~\cite{barham2003xen,rosenblum1999vmware,smith2005architecture}
or lightweight virtualization techniques like Linux 
containers~\cite{felter2014updated,merkel2014docker}. This model allows each user to 
have her own environment with its own package manager, and it has been extremely 
successful for servers at cloud data centers.  VMs typically have 
near-native compute performance, but there are still major performance issues with
low-level HPC network drivers.  VMs are not well supported on many
non-Linux operating systems, an issue for the lightweight 
kernels of bleeding-edge Blue Gene/Q and Cray machines. Security
is another major concern: a profusion of VMs makes mandatory patching
difficult when user's environments differ significantly from those of the facility.

\begin{table*}\centering
\begin{tabular}{|l|l|}
\hline
Site           & Naming Convention \\
\hline
\hline
LLNL       & {\tt / usr / global / tools / \$arch / \$package / \$version} \\
           & {\tt / usr / local~ / tools / \$package-\$compiler-\$build-\$version } \\
\hline
Oak Ridge~\cite{jones+:cug08}  & {\tt / \$arch / \$package / \$version / \$build} \\
\hline
TACC/Lmod~\cite{mclay:lmod-tutorial}& {\tt / \$compiler-\$comp\_version / \$mpi / \$mpi\_version / \$package / \$version} \\
\hline
\hline
Spack default                  & {\tt / \$arch / \$compiler-\$comp\_version / \$package-\$version-\$options-\$hash} \\
\hline
\end{tabular}
\caption{
	Software organization of various HPC sites.
	\label{tab:naming-conventions}
}
\end{table*}

\paragraph{Manual and Semi-automated Installation}

To cope with software diversity, many HPC sites use a combination of existing
packages managers and either manual or semi-automated software installation.  
For the baseline OS, it is common for sites to maintain traditional binary
packages using the vendor's package manager.  LLNL maintains a Linux 
distribution, CHAOS~\cite{chaos} for this purpose.

For more specific builds, many sites adhere to detailed naming conventions
that encode information in filesystem paths. A sampling of conventions
at several sites is shown in Table~\ref{tab:naming-conventions}. At LLNL, the
APT package manager is used for installs in the {\tt /usr/local/tools} 
filesystem, and {\tt /usr/global/tools} is used for manual installs.
ORNL uses hand installs but adheres to a strict scripting structure
that can be used to reproduce each build~\cite{jones+:cug08}.
TACC relies heavily on locally maintained RPMs for its software installs.

Looking again at the conventions in Table~\ref{tab:naming-conventions},
we see that most sites use some combination of architecture, compiler version,
package name, package version, and a custom (up to the author, sometimes
encoded) build identifier.  TACC and many other sites also include the MPI
version in the path. None of these naming conventions is defined for
the entire configuration space, and none defines a way to create new versions
from arbitrary combinations of dependencies.  MPI is explicitly called out,
but it is only one of many dependencies that go into a build.
Sites have made a conscious choice to divide the directory hierarchy
according to the most common software packages, but they ignore the rest.
Frequently, desired configurations do not fit the local convention, and
staff must install them in nonstandard locations.

\paragraph{Environment Modules and RPATHs}

Diverse software versions not only present problems for build and installation; 
they also complicate the runtime environment. When run, an executable
must determine the location of its dependency libraries, or it will fail to run.
This is typically not a problem for statically linked executables, but modern
operating systems make extensive use of dynamic linking, which has many 
benefits for software tools and maintenance.
By default, the dynamic loader on most systems is configured to search only
system library paths such as {\tt /lib}, {\tt /usr/lib}, and
{\tt /usr/local/lib}.  If binaries are installed to other locations, the
{\it user} who runs the program must typically add dependency library paths to
{\tt LD\_LIBRARY\_PATH} (or a similar environment variable on other OSes).
Most HPC sites address this problem using {\it environment modules}, which allow
users to dynamically ``load'' and ``unload'' settings for search paths like
{\tt LD\_LIBRARY\_PATH}, {\tt PATH}, {\tt MANPATH}, etc.

Environment modules date back to 1991, and there are many implementations 
available~\cite{dotkit,furlani+:lisa91,furlani+:lisa96,mclay:lmod,mclay:lmod-tutorial}.
The most advanced implementations, such as Lmod~\cite{mclay:lmod,mclay:lmod-tutorial},
provide software hierarchies similar to the naming conventions in 
Table\ref{tab:naming-conventions}, and they allow users to quickly load a software stack
if they know which one is required.  TACC expects the user to run with the same
environment that she built with~\cite{mclay:lmod-tutorial}.  But, at LC, the user is
frequently not the builder of the application, and may not know the particular
incantation required to load the its environment. Moreover, users frequently run more 
than one application in the same job, each built by a separate team, and each with
its own environment. Most LC users are used to running programs with
specific paths; only 10\% use environment modules extensively.

The alternative to per-user environment settings is to embed library search paths
in installed binaries at compile time. When set this way, the search path is called
an {\tt RPATH}. {\tt RPATHs} and environment modules are not mutually exclusive.
Modules can still be used to set {\tt MANPATH}, {\tt PATH} and variables, but
adding {\tt RPATHs} ensures that binaries will run correctly regardless of whether
the right module is loaded. LC installs software with both {\tt RPATHs} and
{\tt dotkit}~\cite{dotkit} modules.

\paragraph{Modern Package Managers}

Recently, a number of package managers have been developed to manage 
multi-configuration builds.
%
Nix~\cite{dolstra+:icfp08,dolstra+:lisa04}
is a package manager and an OS distribution that supports the installation of
arbitrarily many software configurations.  As at most HPC sites, each package
in Nix installed in a unique prefix, but in Nix, there is no human-readable
naming convention.  Instead, Nix determines the prefix by hashing the package
file and along with its dependencies.

Hashdist~\cite{hashdist} is a meta-build system and package manager for HPC.
Like Nix, it uses cryptographic versioning, and it archives installations
in a git repository so that they can be recalled on demand. 
%
Both Nix and hashdist use {\tt RPATHs} in their packages to ensure that
libraries are found correctly. 

The EasyBuild~\cite{hoste+:pyhpc12} tool is in production use at
the University of Ghent.  It allows multiple versions to be installed
at once.  Rather than setting {\tt RPATHs}, it auto-generates module files
to manage each package's environment, and it is closely coupled with 
Lmod~\cite{geimer+:hust14}.  EasyBuild groups the compiler, MPI, FFT, and
BLAS libraries together in a {\it toolchain}, which can be combined with
any package file. This provides some degree of composability and
separates compiler flags and MPI concerns from client packages.

Smithy~\cite{digirolamo:smithy} is a similar tool in use at ORNL. It
can generate module files, but it does not provide any 
automated dependency management; it only checks whether a package's
prerequisites have already been installed by the user.

\paragraph{Gaps in the state of the art}
The cryptographic versioning of Nix and Hashdist is very flexible.  It versions
the package {\it and} its dependency configuration, and can represent any
configuration. However, there is no way for users to navigate or query
software. The systems are in a sense ``write-only'':
Nix does not even offer a way to ask for a package's dependencies.
%
EasyBuild and Smithy generate environment modules, which allow querying
through their own interface. Naming in existing module systems, however, cannot
handle combinatoric versions. The Lmod authors call this the
``matrix problem``~\cite{mclay:lmod-tutorial}.

%EasyBuild's attempts to version groups dependencies by adding versions to
%toolchains, but the naming is difficult to understand.
%. Compiler, MPI, and 
%some library versions are lumped together, but the results is cryptic, e.g.
%{\tt goolf} stands for ``gcc, openmpi, openblas, ScaLAPACK, FFTW'', and
%its version is meaningless.

Existing tools do not enforce {\tt RPATHs}, leaving them to package authors.
This may lead to erroneous runs.

The main limitation of existing tools is the lack of build {\it composability}.
The full set of package versions is combinatoric, and realizing an arbitrary
combination of compiler, MPI version, build options, and dependency versions
requires tediously modifying many package files.
%
Indeed, the number of package files required for existing systems scales with
the number of version {\it combinations}, not the number of packages.
This quickly becomes unmanageable.  As an example, the very flexible EasyBuild
system has over 3,300 files for several permutations of only 600 packages. 
Other systems have similar issues. A mechanism is needed to {\it parameterize}
packages so that new builds can be {\it composed} on the fly.











