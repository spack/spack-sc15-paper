%!TEX root = spack-paper-sc15.tex

\section{Common Practice}
\label{sec:motivation}

\paragraph{Meta-build Systems}
{\it Meta-build systems} such as Contractor, WAF, and
MixDown~\cite{amundson:contractor,epperly+:mixdown,epperly+:mixdown-report,nagy:waf} are
related to package managers, but they focus on ensuring that a single
package builds with its dependencies.  MixDown notably provides excellent
features for ensuring consistent compiler flags in a build.
However, these systems do not provide facilities to manage
large package repositories or combinatorial versioning.

\paragraph{Traditional package managers}
Package managers automate the installation of complex sets of software packages.
{\it Binary package managers} such as RPM, yum, and
APT~\cite{foster+:rpm03,silva:apt01,yum} are integrated with most
OS distributions, and they are used to ensure that dependencies
are installed before packages that require them.
These tools largely solve the problem of managing a {\it single} software
stack, which works well for the baseline OS and drivers, which are
common to all applications on a system.
These tools assume that each package only has a single version
and most of the tools install packages in a single, inflexible location.
To install multiple configurations, the user must create a custom, combinatorial
naming scheme to avoid conflicts. They typically require root
privileges and do not optimize for specific hardware.

{\it Port systems} such as Gentoo, BSD Ports, MacPorts, and
Homebrew~\cite{bsdports,groffen:gentoo-prefix,homebrew,macports,thiruvathukal:gentoo04}
build packages from source instead of installing from a pre-built binary.
Most port systems suffer from
the same versioning and naming issues as traditional package managers.
Some allow multiple versions to be installed in the same
prefix~\cite{groffen:gentoo-prefix}, but again the burden is on package
creators to manage conflicts. This burden effectively restricts installations
to a few configurations.


\paragraph{Virtual Machines and Containers}

Packaging problems arise in HPC because a supercomputer's hardware, OS, and
filesystem are shared by many users with different requirements.  The classic
solution to this problem is to use virtual
machines (VMs)~\cite{barham2003xen,rosenblum1999vmware,smith2005architecture}
or lightweight virtualization techniques like Linux
containers~\cite{felter2014updated,merkel2014docker}. This model allows each
 user to have a personalized environment with its own package manager, and it
has been extremely successful for servers at cloud data centers. VMs typically
have near-native compute performance but low-level HPC network drivers still
exhibit major performance issues. VMs are not well supported on many
non-Linux operating systems, an issue for the lightweight
kernels of bleeding-edge Blue Gene/Q and Cray machines.
Finally, each VM still uses a traditional package manager,
so running many configurations still requires a large number of VMs.
For facilities, this is a security concern, as a profusion of VMs makes
mandatory patching difficult.  It is also tedious for users to manage a large
number of VM environments.

\begin{table*}\centering
\begin{tabular}{|l|l|}
\hline
Site           & Naming Convention \\
\hline
\hline
LLNL       & {\tt / usr / global / tools / \$arch / \$package / \$version} \\
           & {\tt / usr / local~ / tools / \$package-\$compiler-\$build-\$version } \\
\hline
Oak Ridge~\cite{jones+:cug08}  & {\tt / \$arch / \$package / \$version / \$build} \\
\hline
TACC/Lmod~\cite{mclay:lmod-tutorial}& {\tt / \$compiler-\$comp\_version / \$mpi / \$mpi\_version / \$package / \$version} \\
\hline
\hline
Spack default                  & {\tt / \$arch / \$compiler-\$comp\_version / \$package-\$version-\$options-\$hash} \\
\hline
\end{tabular}
\caption{
	Software organization of various HPC sites.
	\label{tab:naming-conventions}
}
\end{table*}

\paragraph{Manual and Semi-automated Installation}

To cope with software diversity, many HPC sites use a combination of existing
package managers and either manual or semi-automated installation.
For the baseline OS, many sites maintain traditional binary
packages using the vendor's package manager. LLNL maintains a Linux
distribution, CHAOS~\cite{chaos} for this purpose, which is managed using RPM.
%
For custom builds, many sites adhere to detailed naming conventions
that encode information in file system paths.
Table~\ref{tab:naming-conventions} shows several sites' conventions.
LLNL uses the APT package manager for installs
in the {\tt /usr/local/tools} file system and {\tt /usr/global/tools}
for manual installs.
ORNL uses hand installs but adheres to strict scripting conventions
to reproduce each build~\cite{jones+:cug08}.
TACC relies heavily on locally maintained RPMs.

From the conventions in Table~\ref{tab:naming-conventions},
we see that most sites use some combination of architecture, compiler version,
package name, package version, and a custom (up to the author, sometimes
encoded) build identifier.  TACC and many other sites also explicitly
include the MPI version in the path. MPI is explicitly called out
because it is one of the most common software packages for HPC.
However, it is only one of many dependencies that go into a build.
None of these naming conventions covers the entire configuration
space, and none has a way to represent, e.g., two builds that are identical
save for the version of a particular dependency library.  In our experience
at LLNL, naming conventions like these have not succeeded because
users want more configurations than we can represent with a practical
directory hierarchy. Staff frequently install nonconforming packages
in nonstandard locations with ambiguous names.

\paragraph{Environment Modules and RPATHs}\label{sec:env-rpath}

Diverse software versions not only present problems for build and installation;
they also complicate the runtime environment. When launched, an executable
must determine the location of its dependency libraries, or it will not run.
Even worse, it may find the wrong dependencies and subtly produce incorrect results.
Statically linked binaries do not have this issue, but modern
operating systems make extensive use of dynamic linking.
By default, the dynamic loader on most systems is configured to search only
system library paths such as {\tt /lib}, {\tt /usr/lib}, and
{\tt /usr/local/lib}.  If binaries are installed in other locations, the
{\it user} who runs the program must typically add dependency library paths to
{\tt LD\_LIBRARY\_PATH} (or a similar environment variable) so that the loader
can find them.  The user is often not the same person who installed the binary,
so determining what paths to add can be difficult, especially for inexperienced users.

Many HPC sites address this problem using {\it environment modules}, which
allow users to ``load'' and ``unload'' such settings dynamically using simple
commands. Environment modules emerged in 1991, and there are many implementations~\cite{dotkit,furlani+:lisa91,furlani+:lisa96,mclay:lmod,mclay:lmod-tutorial}.
The most advanced of these, such as Lmod~\cite{mclay:lmod,mclay:lmod-tutorial},
provide software hierarchies similar to the naming conventions in
Table~\ref{tab:naming-conventions}. They allow users to load a software stack
quickly if they know which one is required.

The alternative to per-user environment settings is to embed library search
paths in installed binaries at compile time. When set this way, the search
path is called an {\tt RPATH}. {\tt RPATHs} and environment modules are not
mutually exclusive. Modules can still be used to set {\tt MANPATH}, {\tt PATH},
etc., while adding {\tt RPATHs} ensures that binaries run correctly
regardless of whether the right module is loaded. LC installs software with
both {\tt RPATHs} and {\tt dotkit}~\cite{dotkit} modules.

\paragraph{Modern Package Managers}

Recently, a number of HPC package managers have emerged that manage
multi-configuration builds.
%
Smithy~\cite{digirolamo:smithy} is an installation tool in use at ORNL. It
can generate module files, but it does not provide any
automated dependency management; it only checks whether a package's
prerequisites have already been installed by the user.

Nix~\cite{dolstra+:icfp08,dolstra+:lisa04}
is a package manager and an OS distribution that supports the installation of
arbitrarily many software configurations.  As at most HPC sites, each package
in Nix is installed in a unique prefix, but Nix does not have a human-readable
naming convention.  Instead, Nix determines the prefix by hashing the package
file along with its dependencies. Nix package files are written in a custom
functional language designed for packaging.

The EasyBuild~\cite{hoste+:pyhpc12} tool is in production use at
the University of Ghent and several other sites.  It allows multiple versions
to be installed at once.  Rather than setting {\tt RPATHs}, it
auto-generates module files
to manage each package's environment, and it is closely coupled with
Lmod~\cite{geimer+:hust14}.  EasyBuild groups the compiler, MPI, FFT, and
BLAS libraries together in a {\it toolchain} that can be combined with
any package file. This provides some degree of composability and
separates compiler flags and MPI concerns from client packages.

HashDist~\cite{hashdist} is a meta-build system and package manager
for HPC.  Of the existing solutions, it is the most similar to spack.
Like Nix, it uses cryptographic versioning and stores installations in
unique directories.
%
Both Nix and HashDist use {\tt RPATHs} in their packages to ensure that
libraries are found correctly.

\paragraph{Gaps in current practice}
The cryptographic versioning of Nix and HashDist is very flexible. It versions
the package {\it and} its dependency configuration, and can represent any
configuration. However, users cannot navigate or easily query the installed
software.
%The systems are in a sense ``write-only'':
%Nix does not offer a way to view a package's dependencies.
%
EasyBuild and Smithy generate environment modules, and some querying is
possible through the module system.  Naming schemes used in existing module
systems, however, cannot handle combinatorial versions. The Lmod authors
call this the ``matrix problem''~\cite{mclay:lmod-tutorial}.

%EasyBuild's attempts to version groups dependencies by adding versions to
%toolchains, but the naming is difficult to understand.
%. Compiler, MPI, and
%some library versions are lumped together, but the results is cryptic, e.g.
%{\tt goolf} stands for ``gcc, openmpi, openblas, ScaLAPACK, FFTW'', and
%its version is meaningless.

%Existing tools do not enforce {\tt RPATHs}, leaving them to package authors.
%This may lead to erroneous runs.

The main limitation of existing tools is the lack of build {\it composability}.
The full set of package versions is combinatorial, and realizing an arbitrary
combination of compiler, MPI version, build options, and dependency versions
requires tediously modifying many package files.
Indeed, the number of package files required for most existing systems scales
with the number of version {\it combinations}, not the number of packages, which
quickly becomes unmanageable.  As an example, the EasyBuild
system has over 3,300 files for several permutations of only 600 packages.
A slightly different dependency graph requires an entire new package file
hierarchy.  HashDist a more robust support for composition,
but it does not have a first-class parameters for versions, compilers,
or versioned interfaces.
%
HPC sites need better mechanisms to {\it parameterize} packages so that new
builds can be {\it composed} in response to user needs.
%
