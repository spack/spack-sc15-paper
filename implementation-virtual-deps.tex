%!TEX root = spack-paper-sc15.tex

\subsection{Versioned Virtual Dependencies}\label{sec:virtual}

\begin{figure}
	\begin{minted}[numbersep=5pt,
	              fontsize=\scriptsize,
	              frame=lines,
                  framesep=2mm]{python}
 # providers of mpi                       # mpi dependents
 class Mvapich2(Package):                 class Mpileaks(Package):
     provides('mpi@:2.2', when='@1.9')        depends_on('mpi')
     provides('mpi@:3.0', when='@2.0')           ...
     ...                                  class Gerris(Package):
 class Mpich(Package):                        depends_on('mpi@2:')
     provides('mpi@:3', when='@3:')           ...
     provides('mpi@:1', when='@1:')
     ...
\end{minted}
\caption{
	Virtual dependencies.
	\label{fig:virtual-deps}
}
\end{figure}

Many libraries share a common source interface and can be interchanged for
one another in a build.
The archetypal example in HPC is the MPI interface, which has a number of
open source implementations (MPICH, OpenMPI, MVAPICH, etc.) and vendor-specific
implementations. An application that can be built with one MPI can generally be
built with another.  BLAS implementations such as ATLAS, LAPACK-BLAS, MKL, etc.
are another common example.

At LLNL, we frequently need to build tools with many versions of MPI to support
the many different applications that run at our center.
However, MPI has no ABI, so applications cannot simply be re-linked;
they must be recompiled and reinstalled. 
Complicating matters, the MPI interface is versioned, and some
packages need later versions of MPI to run correctly.  Often, MPI
{\it implementation} versions do not correspond obviously to MPI
interface versions, and determining the right version of 
MPI to pair with an application can be tedious and tricky.

To allow rapid composition of libraries by interface, Spack supports
{\it virtual dependencies}.  A virtual dependency is an abstract name
representing a library interface or capability instead of a library 
implementation.  Packages that need this capability need not depend on 
a specific implementation; they can depend on the virtual name, for which
the user or Spack can select an an implementation at build time.
Other package managers support the notion of virtual dependencies, but Spack
adds {\it versioning} to its interfaces.  This allows concepts like MPI
interface versions and BLAS levels to be represented directly, and Spack can
handle the details of managing complex constraints while the user focuses on 
composing software.

Figure~\ref{fig:virtual-deps} shows how packages provide
virtual interfaces in Spack.  The spec syntax is used to concisely
associate ranges of {\tt mpi} versions for the {\tt mvapich2} and {\tt mpich} packages.
The {\tt mpileaks} package requires {\tt mpi}, but it does not constrain the version.
Any version of {\tt mvapich2} or {\tt mpich} could be used to to satisfy the {\tt mpi}
constraint. The Gerris CFD library, however, needs MPI version 2 or higher.  So any
version except {\tt mpich} 1.x could be used to satisfy the constrained dependency.


%{\bf Dependencies.}
%Lines 6 and 7 in the table show the key feature that enables Spack's flexibility.
%The user can specify all of the above information not only for the package being
%installed, but also for its {\it dependencies}.  To do this, the user needs only supply 
%\verb|^| and the dependency's name.  If we need to build a new version with a specific
%version of {\tt mvapich2}, we can simply add, e.g., \verb|^mvapich2@1.9|
%to the spec, and it will build with that MPI version instead of the default.
%This can also be done for multiple libraries in the same spec (line 7).  



