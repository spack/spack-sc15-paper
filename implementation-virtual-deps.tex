%!TEX root = paper.tex


\subsection{Versioned Virtual Dependencies}
	\todo{.5 page}


{\bf Dependencies.}
Lines 6 and 7 in the table show the key feature that enables Spack's flexibility.
The user can specify all of the above information not only for the package being
installed, but also for its {\it dependencies}.  To do this, the user needs only supply 
\verb|^| and the dependency's name.  If we need to build a new version with a specific
version of {\tt mvapich2}, we can simply add, e.g., \verb|^mvapich2@1.9|
to the spec, and it will build with that MPI version instead of the default.
This can also be done for multiple libraries in the same spec (line 7).  


\subsubsection{Querying installed packages}
