%!TEX root = paper.tex

Large HPC centers spend considerable time supporting software for thousands of users, but the complexity of HPC software is quickly outpacing the capabilities of existing management tools. Scientific applications require specific versions of many compilers, MPI versions, and dependency libraries.  This makes a single software stack infeasible, but managing many configurations is difficult because the configuration space is exponential in size.
%
Spack is a package management tool used at Lawrence Livermore National Laboratory (LLNL) to manage software complexity. It allows many different software configurations to build on demand and to coexist on the same system. Spack provides a novel, recursive specification syntax to manage combinatoric builds of packages and dependencies.  It ensures that installed packages find their runtime dependencies, {\it regardless of the environment}. We show through real-world use cases that Spack supports a diverse and demanding user base, bringing order to the chaos of HPC software management.
