%!TEX root = spack-sc15.tex

\subsection{User and Site Policies}
\label{sec:usecase-policy}

Spack makes it easy to create and to organize package installations,
but it must also be easy for end-users to find and use packages on an HPC
system. Different end-users have different expectations about how packages
should be built and installed, and at LLNL those expectations are shaped
by years of site policies, personal preferences, and lingering legacy
decisions originally made on a DEC VAX.
Spack provides mechanisms that allow users to customize it for the needs
of particular sites and users.

%Rather than try to dictate these expectations,
%Spack allows both end-users and package maintainers to create custom policies dictating
%how packages are built and installed at their site.

\subsubsection{Package Views}
\label{sec:package-views}

While Spack can easily create many installations of a package like 
{\tt mpileaks}, end-users may find Spack's 
directory layout confusing, and they may not be able to find libraries 
intuitively. Spack installs packages into paths based on concretized specs,
which is ideal for maintaining multiple package installations. However, 
end-users would find these paths difficult to navigate. 
For example, an {\tt mpileaks} installation prefix might be:
%
\begin{minted}[fontsize=\small]{bash}
spack/opt/linux-x86_64/gcc-4.9.2/mpileaks-1.0-db465029
\end{minted}

As discussed in Section~\ref{sec:envmodule},
environment modules are commonly used on HPC systems to solve this problem,
and Spack allows the package author to create environment modules 
automatically for packages that Spack installs.
Even when modules are available, many users still
navigate the file system to access packages.

Spack allows the creation of {\it views}, which are symbolic-link based directory 
layouts of packages. Views provide a human-readable directory layout that
can be adapted to match legacy directory layouts on which users rely. For 
example, the above {\tt mpileaks} package may have a view that creates a link 
in {\tt /opt/mpileaks-1.0-openmpi} to the Spack installation of {\tt mpileaks}.
The same package install may be referenced by multiple links and views,
so the above package could also be linked from a more generic {\tt /opt/mpileaks-openmpi}
link. This reuse supports users who always want to use the latest version.
Views can also be used to create symbolic links to specific executables or libraries in an install,
so a Spack-built {\tt gcc@4.9.2} install may have a view that creates links from
{\tt /bin/gcc49} and {\tt /bin/g++49} to the appropriate {\tt gcc} and {\tt g++} executables.

Views are configured using configuration files, which can be set up at the site or
user level.
For each package or set of packages, the configuration file contains rules
that describe the links that should point into that package.
The link names can be parameterized.
For example, the above {\tt mpileaks} symbolic link might have been created by a rule like:
%
\begin{minted}[fontsize=\small]{bash}
    /opt/${PACKAGE}-${VERSION}-${MPINAME}
\end{minted}
%
On installation and removal,
links are automatically created, deleted, or updated according to these rules.

Spack's views are a projection from points in a high-dimensional space
(concretized specs, which fully specify all parameters) to points
in a lower-dimensional space
(link names, which may only contain a few parameters).
Several installations may map to the same link.
For example, the above {\tt mpileaks} link could point to an {\tt mpileaks} compiled with
{\tt gcc} or {\tt icc}---the compiler parameter is not part of the link.
To keep  package installations consistent and reproducible,
Spack has a well-defined mechanism for resolving conflicting links;
it uses a combination of internal default policies and user- or site-defined
policies to define an {\it order} of preference for different parameters.
By default, Spack prefers newer versions of packages compiled with newer compilers
to older packages built with older compilers. It has a well-defined, but not
necessarily meaningful, order of preference for deciding between MPI
implementations and different compilers.
The default policies can be overridden in configuration files, by either users
or by sites. For example, at one site users may typically use the Intel compiler,
but some users also use the system's default {\tt gcc@4.4.7}.
These preferences could be stated by adding:
%
\begin{minted}[fontsize=\small]{bash}
    compiler_order = icc,gcc@4.4.7
\end{minted}
%
to the site's configuration file, which would cause the ambiguous
{\tt mpileaks} link to point to an installation compiled with {\tt icc}.
Any compiler not in the {\tt compiler\_order} setting is less 
preferred than those explicitly provided.
%
In a similar manner, Spack can be configured to give specific package
configurations priority over others, which can be useful with a new, unstable 
and untested version.

\subsubsection{External Package Repositories}
By default, Spack stores its package files in a mainline repository
that is present when users
first run Spack.  At many sites, packages may build sensitive,
proprietary software, or they
may have patches that are not useful outside of a certain company or
organization.  Putting this type of code back into a public repository
does not often make sense and, if it makes the
mainline less stable, it can actually make sharing code between sites more difficult.

To support our own private packages, and to support those of LLNL code teams,
Spack allows the creation of site-specific variants of packages.
Users can specify additional search directories for finding additional 
{\tt Package} classes via configuration files.
%
The additional packages are like the {\tt mpileaks} package shown in 
Figure~\ref{fig:mpileaks}.  However, the extension packages can extend from not
only {\tt Package}, but also any of Spack's built-in packages.
Custom packages can inherit from and replace Spack's default
packages, so other sites can either tweak or completely replace Spack's
build recipes. For example, a site can write a local Python class that 
inherits from Spack's base class. The local class may simply add configure 
flags to Spack's version, while leaving the dependencies and most of the 
build instructions from its parent class. For reproducibility, Spack also 
tracks the {\tt Package} class that drove a specific build.
